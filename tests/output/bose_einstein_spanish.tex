\documentclass{article}

\usepackage{amsmath}
\usepackage{wrapfig}
\usepackage{amssymb}
\usepackage{graphicx}
\usepackage{biblatex}
\usepackage{listings}
\usepackage{xcolor}

\definecolor{backcolour}{rgb}{0.95,0.95,0.92}

\lstdefinestyle{mystyle}{
    backgroundcolor=\color{backcolour},
    basicstyle=\ttfamily\footnotesize,
    breakatwhitespace=false,
    breaklines=true,
    captionpos=b,
    keepspaces=true,
    showspaces=false,
    showstringspaces=false,
    showtabs=false,
    tabsize=2
}

\lstset{style=mystyle}

\begin{document}

  \section{Condensado de Bose-Einstein}
  
  Si no podemos modelar un gas con la aproximación clásica
  porque la interacción entre los fermiones y los bosones
  tiene efectos no despreciables, y podemos considerar que
  los únicos grados de libertad son traslacionales, entonces
  el número de ocupación de un estado cuántico, $r$, es
  
  \[\langle n_{r} \rangle = \frac{1}{e^{\beta \left(\epsilon_{r} - \mu \right)} - 1} \]
  
  donde $\beta = \frac{1}{k_{B} T}$  y $k_{B}$  es
  la constante de Boltzmann.
  
  Vale la pena notar que, como los bosones no cumplen el principio
  de exclusion de Pauli y por lo tanto puede haber una infinidad
  de bosones ocupando el mismo estado cuántico, esta función tiende
  a infinito cuando el argumento de la exponencial se aproxima a cero
  y decae rápidamente a temperaturas mayores, pues puede haber
  infinidad de ellos en el mismo estado cuántico.

\end{document} 

