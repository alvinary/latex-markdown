\documentclass{article}

\usepackage{amsmath}
\usepackage{wrapfig}
\usepackage{amssymb}
\usepackage{graphicx}
\usepackage{biblatex}
\usepackage{listings}
\usepackage{xcolor}

\definecolor{backcolour}{rgb}{0.95,0.95,0.92}

\lstdefinestyle{mystyle}{
    backgroundcolor=\color{backcolour},
    basicstyle=\ttfamily\footnotesize,
    breakatwhitespace=false,
    breaklines=true,
    captionpos=b,
    keepspaces=true,
    showspaces=false,
    showstringspaces=false,
    showtabs=false,
    tabsize=2
}

\lstset{style=mystyle}

\begin{document}

  \section{Eigenvalues and eigenvectors}
  
  \subsection{Linear independence}
  
  It is easy to prove that eigenspaces are disjoint.
  We said $v$  is an eigenvector of $T$  if
  
  \[Tx = kx \]
  
  for some $k$  in $K$.
  
  \bigbreak
  
  So if $Tx = kx$  and $Ty = k ' y$, $x$  and $y$  are distinct,
  and $y \neq qx$, then $k '$  and $k$  cannot be equal, and the
  two vectors cannot be elements of the same eigenspace.
  
  \bigbreak
  
  Note if $y + qx$  then $T \left(y \right) = T \left(qx \right) = k \left(qx \right) = ky$, and the
  two vectors do belong to the same eigenspace and have
  the same eigenvalue. So 'scaled versions' of the same
  vector belong to the same eigenspace.

\end{document} 

